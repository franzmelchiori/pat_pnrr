\documentclass[a4paper]{article}
\usepackage{geometry}
\usepackage{graphicx}
\usepackage[table]{xcolor}
\usepackage[official]{eurosym}
\usepackage{booktabs}
\usepackage{longtable}
\usepackage{multirow}
\usepackage{pifont}
\usepackage[italian]{babel}
\usepackage{makeidx}

\geometry{a4paper, top=3cm, bottom=3cm, left=3cm, right=3cm, heightrounded}
\definecolor{custom-blue}{RGB}{47,84,150}

\usepackage{fancyhdr} % per personalizzare testatine e piedi
\pagestyle{fancy} % per applicar le impostazioni di fancyhdr
% impostazioni fancy delle testatine:
\renewcommand{\headrule}{}
%% impostazioni fancy dei piedi:
\renewcommand{\footrule}{\vbox to 0pt{\hbox to\headwidth{\dotfill}\vss}}
\renewcommand{\footrulewidth}{1pt}
\lfoot{\rmfamily \footnotesize{PAT-PNRR Digital}\\\sffamily \ding{41} pnrr.digital@.provincia.tn.it}
\cfoot{\footnotesize \rmfamily \today\\v5.0.0}
\rfoot{\footnotesize PAT-MPE\\\sffamily page \thepage\ of \pageref{LastPage}}
\usepackage{lastpage}

\makeindex

\begin{document}
	\begin{center}
		\includegraphics[height=1.25cm]{../relazione_tecnica/pat-mpe_images/logo_next_generation_eu} \hspace{1.25cm} \includegraphics[height=1.25cm]{../relazione_tecnica/pat-mpe_images/logo_repubblica_italiana} \hspace{1.25cm} \includegraphics[height=1.25cm]{../relazione_tecnica/pat-mpe_images/pat_logo}\\
		\vspace{0.5cm}
	\end{center}
	\begin{center}
		\textbf{\textcolor{custom-blue}{Piano Nazionale di Ripresa e Resilienza, Missione 1, Componente 1 (PNRR-M1C1)}}\\
		\textbf{\textcolor{custom-blue}{Investimento 2.2 ``Task force digitalizzazione, monitoraggio e performance''}}\\
		\textbf{\textcolor{custom-blue}{Sub-investimento 2.2.1 ``Assistenza tecnica a livello centrale e locale del PNRR''}}\\
		\textbf{\textcolor{custom-blue}{``Progetto 1000 Esperti''}}\\
		\vspace{4.75cm}
	\end{center}
	\begin{center}
		\textsc{\LARGE{PAT\footnote{Provincia Autonoma di Trento}-MPE}}\\
        \vspace{0.1cm}
        \textsc{\Large{Monitoraggio Procedimenti Edilizi}}\\
		\vspace{0.75cm}
		\textsc{\large{per il} \Large{Comune di Ledro}}\\
		\vspace{0.75cm}
		\textsc{\Large{\today}}\\
        \vspace{0.3cm}
        \textsc{\large{v5.0.0}}\\
		\vspace{4cm}
	\end{center}
	\begin{flushright}
		\includegraphics[height=1.25cm]{../relazione_tecnica/pat-mpe_images/pat-pnrr_logo}\\
		\vspace{0.5cm}
		\textsc{\normalsize{A cura di}}\\
        \textsc{\large{PAT-PNRR Digital}}\\
		\large{Ing. Francesco Melchiori}\normalsize{, Esperto in Tecnologie Digitali}\\
		\vspace{0.5cm}
		\textsc{\normalsize{Struttura di riferimento}}\\
		\textsc{\large{PAT-UMST\footnote{Unità di Missione Strategica} Digitalizzazione e Reti}}\\
		\large{Dott.ssa Cristiana Pretto}\normalsize{, Dirigente Generale}\\
	\end{flushright}
	\clearpage

	\tableofcontents
	\clearpage
	
	Le elaborazioni numeriche e grafiche della presente relazione sono state programmate e pubblicate con Python e LaTeX. Il codice Python, ad esclusione dei dati, è contenuto nel seguente repository online pubblico GitHub.\\
	\begin{figure}[!htbp]
		\begin{center}
			\includegraphics[height=1cm]{../relazione_tecnica/pat-mpe_images/github_logo.png} \textbf{\texttt{github.com/franzmelchiori/pat\_pnrr}} \hspace{0.3cm}
			\includegraphics[height=1cm]{../relazione_tecnica/pat-mpe_images/python_logo.png} \hspace{0.3cm}
			\includegraphics[height=1cm]{../relazione_tecnica/pat-mpe_images/latex_logo.png}\\
		\end{center}
	\end{figure}
	\clearpage

	\section{Introduzione}
	Nel contesto dell'azione del PNRR per la PAT, in particolare per la \texttt{M1C1-Si2.2.1}, è stato organizzato un monitoraggio semestrale a partire dal 2022 di alcuni procedimenti edilizi in carico ai 166 comuni trentini.

	Questa relazione tecnica, in versione sintetica, intende informare un comune sullo stato numerico del suo particolare monitoraggio, ovvero dei dati che ha dichiarato e delle principali analisi elaborate con essi.

	\section{Definizioni}
	\begin{itemize}
		\item \textbf{PdC}: Permessi di Costruire
		\item \textbf{PdC-OV}: Permessi di Costruire Ordinari e in Variante
		\item \textbf{PdS}: Provvedimenti di Sanatoria
		\item \textbf{Durata [gg]}: durata media in giornate, al lordo delle sospensioni, per concludere con un'espressione (positiva o negativa) l'elaborazione tecnica di una tipologia di procedimenti edilizi (es. PdC, PdS)
		\item \textbf{Termine [gg]}: termine mediano di scadenza in giornate, da normativa, per concludere con un'espressione (positiva o negativa) l'elaborazione tecnica di una tipologia di procedimenti edilizi (es. PdC, PdS)
		\item \textbf{Arretrati}: numero totale delle pratiche avviate ma ancora non concluse, la cui durata al netto delle sospensioni ha superato i termini normativi
		\item \textbf{Avviati}: numero totale delle pratiche avviate, concluse o meno, entro od oltre i termini normativi
		\item \textbf{Pressione PdC}: indice delle difficoltà che gravano sull'operatività dell'ufficio tecnico comunale per l'edilizia nell'elaborazione dei PdC: tale indice è in funzione della durata media rapportata al termine massimo e delle pratiche arretrate rapportate con quelle avviate; in seguito si riporta il grafico dell'andamento temporale della pressione dei PdC per il comune di riferimento
		\item \textbf{Pressione PdS}: indice delle difficoltà che gravano sull'operatività dell'ufficio tecnico comunale per l'edilizia nell'elaborazione dei PdS: tale indice è in funzione della durata media rapportata al termine massimo e delle pratiche arretrate rapportate con quelle avviate; in seguito si riporta il grafico dell'andamento temporale della pressione dei PdS per il comune di riferimento
		\item \textbf{Pressione}: indice delle difficoltà che complessivamente gravano sull'operatività dell'ufficio tecnico comunale per l'edilizia nell'elaborazione dei PdC e dei PdS: tale indice è in funzione della Pressione PdC e della Pressione PdS; in seguito si riportano i grafici della pressione fra PdC e PdS e di quella complessiva con il posizionamento del comune di riferimento fra tutti gli altri
		\item \textbf{Elaborazione [ore/settimana]}: ore totali a settimana dedicate, da una o piu' persone ad un comune, per l'elaborazione tecnica di tutti i procedimenti edilizi privati (tra cui i PdC e i PdS)
	\end{itemize}
	\clearpage

	\section{Dati ed analisi}
	\input{../relazione_tecnica/pat_pnrr_performance_tables/pat_pnrr_performance_table_Ledro.tex}
	\begin{center}
		\includegraphics[height=5cm]{../relazione_tecnica/pat_pnrr_performance_charts/pat_pnrr_performance_chart_Ledro.png}
	\end{center}
	\begin{center}
		\includegraphics[height=5cm]{../relazione_tecnica/pat_pnrr_performance_charts/pat_pnrr_performance_organico_chart_Ledro.png}
	\end{center}

\end{document}
